\documentclass{article}
\usepackage[utf8]{inputenc}

\title{AP - lecture 2}
\author{Jannick Drews}
\date{\today}

\usepackage{natbib}
\usepackage{graphicx}

\setlength\parindent{0pt}

\begin{document}

\maketitle
\newpage

\section{Pre-lecture - Equations \& Topics}
Preparation: Chapter 9: Stereo Audio\\
\begin{itemize}
  \item Linear panning\\
      $ x =  \frac{p}{200} + 0.5$\\
        Where $p$ is the panning value, panning is usually normalized from $-100 - 100$, to $0 - 1$. $ p = p_N$
  \item Square-law panning\\
      $ Amp_r = \sqrt{x} $\\
      $ Amp_l = \sqrt{(1-x)} $
  \item Sine-law panning\\
      $ Amp_r = sin(x \cdot \frac{\pi}{2}) $\\
      $ Amp_l = sin((1-x) \cdot \frac{\pi}{2}) $
  \item MS Encoding\\
      $ M = \frac{1}{2}(L + R) $\\
      $ S = \frac{1}{2}(L - R) $
  \item MS Decoding\\
      $ L = M + S = \frac{1}{2}(L + R) + \frac{1}{2}(L - R) = \frac{1}{2} \cdot L + \frac{1}{2} \cdot L $\\
        $ R = M - S =\frac{1}{2}(L + R)  - \frac{1}{2}(L - R) = \frac{1}{2} \cdot R + \frac{1}{2} \cdot R  $
    \item Stereo Image widening\\
        $ width \in [0 , 2] $\\
        $newSides = width \cdot origSides$\\
        $newMid = (2 - width) \cdot origMid $
\end{itemize}

\section{Lecture}

\section{Knowledge}
\begin{itemize}
    \item Explain the basis for representing signals in space.\\
        \textit{.}
    \item Give examples of how signals can be spatialized in practice\\
        \textit{.}
    \item Explain and give examples of amplitude panning.\\
        \textit{Linear panning, Square-law panning, sine-law panning.}
\end{itemize}

\section{Important notes}


\end{document}
