\documentclass{article}
\usepackage[utf8]{inputenc}

\title{DAE - Lecture 1}
\author{Jannick Drews}
\date{\today}

\usepackage{natbib}
\usepackage{graphicx}

\setlength\parindent{0pt}

\begin{document}

\maketitle
\newpage

\section{Pre-lecture}
Preparation: \\
\begin{itemize}
  \item \textbf{.}\\
    \textit{.}
\end{itemize}

\section{Lecture}
Research question:
\begin{itemize}
    \item Does informational context affect perceived music?
\end{itemize}

Variables:
\begin{itemize}
    \item Independent:\\
        Visual or No visual
    \item Dependent:\\
        Survey answers, skill, quality and pleasantness.(Ordinal)
\end{itemize}

Concerns:
\begin{itemize}
    \item Subjective answers on the survery.
    \item Subjective answers on music.
    \item Following instructions.
    \item Sound quality uncontrollable.
    \item If the instruments were really made of ice(for the visual)
    \item Musical knowledge.
    \item Musical preference.
\end{itemize}

\section{Experiments}
Experimental vs correlational research\\
Failing to prove a theory, means we can build on it confidently.\\
From previous experiment, make H0 and H1.\\
\underline{An H0 would be:} Informational context does not affect how music is perceived.\\
\underline{An H1 would be:} Informational context affects how music is perceived.\medskip

\textit{See Hypothesis.pdf}\medskip

Measurement: Definite magnitude of a quantity, with comparison to an agreed value.\\

Be specific with hypothesis, compound variables, expressiveness of data.

\newpage
\section{Knowledge}
From worst to best in complexity/expressiveness of data:
\begin{itemize}
    \item Nominal. (From latin name)\\
        \textit{names or numbers}
  \item Ordinal\\
    \textit{Order but no measure.}
  \item Interval\\
    \textit{Measurable difference.}
  \item Ratio\\
    \textit{Ruler}
\end{itemize}
\includegraphics[width=\textwidth]{data.png}

\section{Important notes}


\end{document}
