\documentclass{article}
\usepackage[utf8]{inputenc}

\title{Lecture 2 - DAE}
\author{Jannick Drews}
\date{\today}

\usepackage{natbib}
\usepackage{graphicx}

\setlength\parindent{0pt}

\begin{document}

\maketitle
\newpage

\section{Pre-lecture}
Preparation: 100 pages\\
\begin{itemize}
  \item \textbf{.}\\
    \textit{.}
\end{itemize}

\section{Lecture}
\begin{enumerate}
    \item Clear idea of what we want to measure (Define what it means)
    \item How to measure(Compare values to an agreed upon standard unit)
    \item What can go wrong\\
        \textit{Random errors, systematic errors, biases, random errors are unavoidable apparently.}
    \item Cause $\rightarrow$ Consequence\\Confounding varaible!
    \item Levels of measurement
    \item Methods
\end{enumerate}

\section{Importance for measurements}
%Reliability is precision.\\
\textbf{Reliability} is "The ability of the measure to produce the same results under the same conditions"\\
%Vailidy is how relevant is it in the context.(Young)\\
\textbf{Validity} is "Whether an instrument measures what its set out to measure."\\
- (Content)Internal: If the effect is not due to your cause.\\
- (Ecological)External: E.g. Making sure the target-group is in their natural habitat???\\
\textbf{Importance} Are we doing research that actually matters?\\
- Someone would benefit.\\
- Problem should not be already solved.\\
- Solvable within 4 months.\\

\section{Levels of measurement}
\begin{itemize}
    \item Nominal\\
        \textit{Labels, Categories: E.g. Football player numbers (categories, of purpose(labels)\\Presentation: Measure of how much the labels perform something, bar-graph)}
    \item Ordinal\\
        \textit{Order of data: E.g. Histograms}
    \item Interval\\
        \textit{Measureable difference: E.g. Degrees celcius.\\No reasonable starting point}
    \item Ratio\\
        \textit{Measureable minimum and maximum}
\end{itemize}

\section{Methods}
Triangulation.\\
Use more than 1 method\\
Likert scales.\\
Observation\\
Interviews\\
Whatever comes to mind\\
Likert item.\\

\section{Likert item VS. Likert scale}
Several likert items makes up a likert scale.\\
Never ask one direct question, always ask several of the same content/category but differently, to increase the reliability of the data.\\
Averaging one item is a sin.\\
Averaging several items from same scale, gives reliable answers.\\
Likert scale are sin, but likert item is deadly sin?\\
Be subtle in the negative and positive related questions.\\
A likert scale evaluation should be $> 0.8$ on the excellence scale, and above $> 0.6$ is acceptable.

\section{4th semester}
Tech test of design\\
Usability test of design\\
pre-test, post-test design. (For making sure the effects and causes are the result of the experiment/test)\\

\section{Knowledge}
Measurements should be within range of the measurement tool, e.g. measureing length of a pen with a ruler would give a range of:
$$R1 < length < R2$$
Keeping more than 1 decimal, means you have measurement device that supports up to more precise measures. \\
E.g. a measure of around 103 would be shown as:
$$ 102.5 < length < 103.5 $$
If taking one-third of the measurement:
$$ 102.5 \cdot \frac{1}{3} < length \cdot \frac{1}{3} < 103.5 \cdot \frac{1}{3}$$
\section{Important notes}


\end{document}
