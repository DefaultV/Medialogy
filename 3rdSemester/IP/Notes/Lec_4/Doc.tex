\documentclass{article}
\usepackage[utf8]{inputenc}

\title{Lecture 4 - IP - Neighbourhood Processing}
\author{Jannick Drews}
\date{\today}

\usepackage{natbib}
\usepackage{graphicx}

\setlength\parindent{0pt}

\begin{document}

\maketitle
\newpage

\section{Pre-lecture}
Preparation: Chapter 5 (not 5.2.2 and 5.2.3) + chapter 12 \\

\section{Lecture}
Ranked filters: Ranking the values in the neighbourhood(sort)\\
\subsection{Correlation - EXAM}
Fundamentals for machine-learning (Neural networks)
\subsubsection{1 Dimension}
\textbf{Filter coefficients; Values inside the filter.}\\
$$g(x) = h(x)f(x) = \sum_{i=-n}^n h(i)f(x+i)$$
where $n$ is the distance to the middle of the filter, a filter with a width of 3, has $3 \ge n = 1$, and since filters start from -1, n is equal to 1, since $-1, 0, 1$.\\
This is also because of the position of the anchorpoint, since 0 is in the middle it goes from -1 to 0 to 1, whereas an anchorpoint in the first cell would go from 0 to 1 to 2.\\
If the filter response is out of range(255), we typically normalize the response by adding the filter values together$1+2+1 = 4$ so if the response was 285 it would be $\frac{285}{4}$\\
\subsection{Convolution vs. Correlation}
Convolution is nearly the same, although the filter is mirrored, the last filter cell would modify the first cell in the filter response or output cell.\\

\subsection{Applications of both}
\begin{itemize}
  \item Object detection
  \item Blur image
  \item Remove noise
  \item Edge detection
  \item Morphology(Later)
\end{itemize}

\subsection{2 Dimensions}
Correlation: $$ g(x,y) = \sum_{j=-R}^{R} \sum_{i=-R}^{R} h(i,j)f(x+i, y+j) $$\\
Convolution: $$ g(x,y) = \sum_{j=-R}^{R} \sum_{i=-R}^{R} h(i,j)f(x-i, y-j) $$\\
\subsection{Boundary problem}
Solutions can be:
\begin{itemize}
  \item Zero-padding.
  \item Replication: Copy the cell value besides the padding
\end{itemize}

\subsection{Template matching}
Using another image as a correlation/convolution kernal filter, to receive a filter reponse, where the highest reponse value is the best possible match for the template.\\
Features, is for finding generalized objects.\\

\subsection{Blurring}
Image blurring, also called smoothing kernels, mean filter or low-pass filter. A spatial low-pass filter is a filter with the same values. A gaussian filter is like a sombrero hat, so the closer to the kernel center the bigger the values(there could also be potentiall big values on the edges of the kernel).\\

\section{Exercises}
\underline{Exercise} $\frac{1}{3}$
Homogenous blur, gaussian blur, bilateral filter, median filter, discuss/show the effects of different sizes.\\ Improve quality of enchanceme.

\underline{Exercise} $\frac{2}{3}$

\underline{Exercise} $\frac{3}{3}$


\section{Knowledge}
\begin{itemize}
  \item Convolution\\
    \textit{.}
  \item Correlation\\
    \textit{.}
  \item template matching\\
    \textit{.}
  \item smoothing filter\\
    \textit{.}
  \item median filter\\
    \textit{.}
\end{itemize}

\section{Important notes}
When the filter is symmetric, then correlation is equal to convolution

\end{document}
