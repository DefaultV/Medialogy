\documentclass{article}
\usepackage[utf8]{inputenc}

\title{ Lecture 6 - IP - notes }
\author{Jannick Drews}
\date{\today}

\usepackage{natbib}
\usepackage{graphicx}

\setlength\parindent{0pt}

\begin{document}

\maketitle
\newpage

\section{Pre-lecture}
Preparation: Read section 5.2.2 and 5.3 (Edge detection)\\

\section{Lecture}
\section{Sobel}
Based on First-order derivative
\begin{itemize}
  \item Noise reduction\\
    \textit{Median/Mean filter}
  \item Edge enhancement\\
    \textit{Calculate candidates for the edges}
  \item Edge localization\\
    \textit{Decide which edge candidates to keep}
\end{itemize}
$$ g_x \approx f(x+1, y) - f(x-1, y)$$
$$ g_y \approx f(x, y+1) - f(x, y-1)$$
Gradient vector: $\vec{g} = [g_x, g_y]$\\
Magnitude: $g_m = \sqrt{g_x^2 + g_y^2} \approx |g_x|+|g_y|$\\

\section{Canny}
Based on groups of edges
\begin{itemize}
  \item Noise reduction\\
    \textit{2d Gaussian used for smoothing}
  \item Edge enhancement\\
    \textit{Magnitude of gradient vector}
  \item Edge localization\\
    \textit{Thinning edges using non-maximal suppresion;\\No sobel effect, just a single-pixel edge found in the input-image. (Finds direction of the gradient; $tan^{-1}(\frac{g_y}{g_x})$, then if the neighbouring pixels go in same direction, the one with the smallest magnitude is suppressed(0), we're trying to find a local maxima on the direction of the gradient.)}
\end{itemize}
We end up with the thresholding dilemma again, this can be fixed with Hysteresis Thresholding.\\
\section{Hysteresis Thresholding}
Check slides\\
(Basically matches the thresholding with the upper-threshold and sees if there's any pixels alike to support the edge, else suppress it.)
Conclusion:\\
Pros:\\
One pixel wide edges\\
Edges are grouped together (often good for segmentation)\\
Robust against noise!\\

Cons:\\
Complicated to understand and implement\\
Slow\\
Used a lot!\\


\section{Important notes}
(Pre-processing)\\
$Edge \rightarrow ROI \rightarrow Feature \rightarrow fill \rightarrow segment$\\
Edges:\\
Shi-Tomasi method\\
Harris algorithm\\
Scale-space corner detectors(Old SoTA)\\
Term: \textbf{Keypoint}, refer to corners and regions, a point in the image where it doesn't change no matter how much you edit.\\
Getting magnitude and orientation (HOG), SIFT, SURF (basic ones)\\
Feature correspondence, ascertaining which parts of one image correspond to which parts of another image. (Homograhpy mapping (MACHINE-LEARNING)).

\section{Important}
Make a edge detection algorithm\\

\section{Mini-project}
It's all about making your own function.

\end{document}
