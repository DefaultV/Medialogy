\documentclass{article}
\usepackage[utf8]{inputenc}

\title{Lecture 1 notes and knowledge - IP}
\author{jdrews17}
\date{September 2018}

\usepackage{natbib}
\usepackage{graphicx}

\setlength\parindent{0pt}

\begin{document}

\maketitle
\newpage

\section{Pre-lecture}
\begin{itemize}
  \item Read: Chapter 1, 2, 10(browse)
\end{itemize}

\section{Lecture}
\subsection{Pre-face: Introduction}
Deep learning \& AI\\

\subsection{Image processing}
Rotation, scaling, blurring \& remove part of image.\\
Combine graphics with real images.\\
Combine part of one image with another.\\
How to find and follow objects in an image.\medskip \\
Basic image processing: ImageJ.\\
Prerequisite for exam:\\
\begin{itemize}
  \item Exercises solving
  \item Participating in explaining the solved exercises to the rest of the class
  \item Micro and Mini-projects, including their presentations
\end{itemize}
Sensors create electric charge when exposed to light, later converted it to a digital image\\
Motion blur can be a result of the speed of the shutter. The speed of the shutter can also lead to over or under-exposure.\\
Digital images are seen as a discrete function(x,y) $f(x,y)$, whereas an image seen as a continuous funtion would be an analog one(film).\\
ROI(Region of Interest) vs background. The ROI can be found with object tracking or any tracking.

\subsection{Camera}
Optical system(lens) acts as a barrier, allowing only specific rays to reach the sensor(s). It also focuses bundles of rays into single points. Focal point(F) and Focal length(f), Optical center(O), both F and O span the optical axis.\\
Distance from object to lens(g), Distance from lens to where the rays intersect(b).\\
$$\frac{1}{g}+\frac{1}{b} = \frac{1}{f}$$
$$\frac{b}{B} = \frac{g}{G}$$
Aperture can close on a lens, making the rays of light come in at a steeper angle, giving more focus but could potentially increase blur. (Glasses sometimes help the aperture in the eye, if a user is straining their eyes focusing)\\
F.O.V (v), depends on sensor and focal length, the smaller the focal length, the larger the F.O.V.
\subsubsection{Zoom}
Optical zoom, algorithms and focal length + aperture = distance from camera.

\subsubsection{Light}
Having correct lighting conditions can be crucial to computer vision related topics.


\section{Knowledge}
\begin{itemize}
  \item How is an image formed?\\
    Sensors and electric charge reflected onto a 2d-array

  \item How is a pixel represented?\\
    bits and bytes, with different channels.(256 values, greyscale)

  \item Pros and cons of back-lighting?\\
    Makes the object stand out, since the light stems from behind an object, making the object a black silhouette. Con is light-dependent? Pros, very clear object tracking i guess.

  \item Explain the following terms:
    \begin{enumerate}
      \item Focus
      \item Depth-of-field
      \item Zoom
      \item F.O.V
      \item Focal length
      \item Shutter
      \item Aperture
    \end{enumerate}
\end{itemize}

\section{Important notes}
\subsection{Micro-project}
check slide

\end{document}

