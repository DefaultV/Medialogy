\documentclass{article}
\usepackage[utf8]{inputenc}

\title{Lecture 5 - IP notes}
\author{Jannick Drews}
\date{\today}

\usepackage{natbib}
\usepackage{graphicx}

\setlength\parindent{0pt}

\begin{document}

\maketitle
\newpage

\section{Pre-lecture}
Preparation: read chapter 7 - Introduction book\\
\begin{itemize}
  \item \textbf{.}\\
    \textit{.}
\end{itemize}

\section{Lecture}
\subsection{BLOB analysis}

\subsection{Region growing}
Thresholding + Component analysis.\\
Specifying a seed-point and then grow the region to threshold and do some component analysis. 

\subsection{Shapes}
\begin{itemize}
  \item Compactness\\
    $\frac{Area}{Width \cdot Height}$
  \item Circularity\\
    $\frac{Perimeter}{2\sqrt(\pi \cdot Area)}$
  \item Feret's diameter (Longest distance)\\
  \item Orientation of Feret's diameter\\
\end{itemize}

\subsection{Classification}
Trusting specific features more, e.g. weighting them more in an image for classification, we can modify the euclidean distance formula:
$$ D = \sqrt(W_1(x_1 - x_2)^2 + W_2(y_1 - y_2)) $$
This is a very useful option for machie-learning applications. Using classifications, such as shape, center and features thelike, is much more efficent for detecing e.g. humans, since skin-hair etc color is non-vital for this sort of implementation.\\
Using ratios for feature matching is much better, e.g. $\frac{w}{h}$ instead of w and h seperately. This also goes for normalization (ratio between features).
Normalization: $$x' = \frac{x-x_{min}}{x_{max}-x_{min}}$$

\section{Machine-learning - k Nearest Neighbour}
The nearest neighbour in the form of a 2d graph, given the different positions of classes with features f1 and f2, a specific class will be defined as a class of the same type as the closest neighbour. K stands for how many neighbours are taken into consideration (consideration will be a value of how many neighbours to compare to, e.g. (two triangles and a circle are neighbours))
Probability of a class is given by: $$ p(C_j|x) = \frac{k_j}{K} $$
So, 2 circles and a triangle neighbour, where $ C_j \epsilon (Circle, Triangle), K=3$\\
$p(Circle|x) = \frac{2}{3}$
$p(Triangle|x) = \frac{1}{3}$

\section{Knowledge}
\begin{itemize}
  \item .\\
    \textit{.}
\end{itemize}

\section{Important notes}
\section{Exercise}
%Slide 13, 14, 18, 32, 33, 43, 47, 48, 49, 50

\end{document}
