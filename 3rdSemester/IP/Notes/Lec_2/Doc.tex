\documentclass{article}
\usepackage[utf8]{inputenc}

\title{Lecture 2 notes and knowledge - IP}
\author{jdrews17}
\date{September 2018}

\usepackage{natbib}
\usepackage{graphicx}

\setlength\parindent{0pt}

\begin{document}

\maketitle
\newpage

\section{Pre-lecture}
\begin{itemize}
  \item .
\end{itemize}

\section{Lecture}
\subsection{Pre-face: Introduction}

\subsection{Color definition.}
Bayer pattern, is used on sensors to get RGB channels correctly, these are scattered equally in a grid.
The mean intensity can be found by: $$I = \frac{(R+G+B)}{3} $$
Converting a proper grayscale can be done by: $$ I = W_R \cdot R + W_G \cdot G + W_B \cdot B \ where \ sum(W_X) = 1$$
Chromaticity plane, triangulating from a cube we can find the plane by having $R = 1,\ G = 1,\ B = 1$. It represents the true color, here we don't talk about intesity but color only.\\
\section{Knowledge}
\begin{itemize}
  \item What does Point Processing mean?
   \item Describe Brightness and Contrast
   \item Describe greylevel mapping and how it relates to Brightness and Contrast
   \item What is a histogram?\\
     \textit{A plot of how much of a value occours throughout a specific scenario.}
   \item How can a histogram be used to choose the greylevel mapping?\\
     \textit{By utilizing a histogram of the intesity/brightness of an imag, we can average out a picture and get a proper visible image from a potential previous too dark or bright of an image}
   \item What is histogram stretching?\\
     \textit{By "stretching" the histogram, we can achieve better lighting-conditions in an image.}
   \item What is thresholding and how is it related to a histogram and to segmentation?\\
     \textit{By thresholding, we can filter out specific colors or scales we want to ROI. Utilizing histogram stretching we can better find a proper lighting-environment than if controlled by external lights.}
   \item What is the difference between Achromatic and Chromatic?\\
     \textit{Achromatic is intensity of the light.\\Chromatic is light waves and the visual range.}
   \item What is the difference between Subtractive Color and Additive Color?\\
     \textit{Additive color gives a white color value when the rest are added up.\\Subtractive gives a black color value (Like the sun).}
   \item Describe the three different color spaces (RGB, rgI, HSI)\\
     \textit{RGB; Red Green blue, values goes from 0 to 255.\\HSI; Hue saturation and intensity.}
   \item What are their characteristics and where are they used?
\end{itemize}

\section{Important notes}



\end{document}

