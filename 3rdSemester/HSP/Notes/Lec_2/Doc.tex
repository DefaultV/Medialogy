\documentclass{article}
\usepackage[utf8]{inputenc}

\title{Lecture 2 notes and knowledge}
\author{jdrews17}
\date{September 2018}

\usepackage{natbib}
\usepackage{graphicx}

\setlength\parindent{0pt}

\begin{document}

\maketitle
\newpage

\section{Pre-lecture}
Preparation:\\
Chapter 2, except diseases and 3 (pages 87-92)
\begin{itemize}
  \item \textbf{what is light?}\\
    \textit{\textbf{Book:} Electromagnetic radiation that sends out particles which we define as \textbf{photons}. The property called 'wavelength' in this radiation defines what kind of electromagnetic radiation it is. Ergo, what we would call light; corresponds to a small slice of wavelengths in the middle of the electromagnetic spectrum. More specifically, from 370nm to 730nm which defines what kind of colour people see. \textbf{Sofia:} A stream of photons, tiny particles that each consist of one quantum of energy which behaves like a wave.}
    \item \textbf{what causes the optic array to vary over time?}\\
      \textit{The changes in lighting condition and objects in a scene, more specifically changes in photons angle of entering the eyes.}
    \item \textbf{how do different parts of the human eye compare to those of a camera?}\\
      \textit{Use brain.}
    \item \textbf{what is acuity?}\\
      \textit{Check notes from: lecture 1; more specifically. It's a measure of how clearly we can see \textbf{fine detail.}}
    \item \textbf{what are photoreceptors and how do they differ in spectral sensitivity?}\\
      \textit{A photoreceptors' main function is to \textbf{transduce light into neural signals}. Located in the back of the eye. There's two different classes of these receptors, the \textbf{Rods} and \textbf{Cones}. The rods are for seeing in the dark, very sensitive to light. The cones are for seeing color and may contribute to finer detail of the vision.\\ From the book: 'Rods: Provide black-and-white vision in dim light. Cones: Provide high-acuity color vision in bright light'. Finally there's the subclasses of the cones; the S/M/L-cones, being sensitive to the corresponding S/M/L-wavelengths of light.}
    \item \textbf{How does dark adaptation work?}\\
      \textit{The sensitivity of the visual system((More specifically; adjusting retinal sensitivity)Adjustment from the sensitivity of photoreceptors so they respond to lower levels of light) automatically adjusts to deal with the mismatch in retinal ganglion cell(RGC) \textbf{spike ranges}, in order to effectively handle the average amount of light striking the retina in any given scene.}
    \item \textbf{What is the relationship between convergence in retinal circuits and light sensitivity?}\\
      \textit{A higher degree of convergence, better supports sensitivity to dim light, because the signals from all the photoreceptors in the circuit are combined by being channeled into a \textbf{single} RGC.\\A lower degree of convergence, better supports visual acuity, because different spatial patterns of light stimulate different photoreceptors, which in turn are sent to \textbf{separate} RGCs.}
    \item \textbf{Explain how a retinal center-surround receptive field works.}\\
      \textit{Stimulation of the center of a receptive field evokes a different response from the RGC than does stimulation of the sorrounding portion of the field. They can be either on-center or off-center.\\On-center receptive fields: increase firing rate when the amount of light striking the center of the cell's receptive field increases relative to light striking the surrounding area.\\Off-center receptive fields: decreases firing rate when the amount of light striking the center of the cell's receptive field decreases relative to the amount of light striking the surrounding area.}
    \item \textbf{How does the retinal fields enhance edges?}\\
      \textit{By utilizing the Center-surround receptive fields that exhibit lateral inhibition. Some of the RGCs will fire with a greater firing rate than the the surrounding RGCs, hence making the area appear brighter. (Very simplified, see p. 69)}
    \item \textbf{How does a simple cell in the primary visual cortex signal its preferred orientation?}\\
      \textit{The orientation that tend to produce the stongest reponse, is determined by flashing bars with various orientations in the receptive field, recording the number of AP (spikes) evoked by each flash, and then calculating the average number of spikes evoked by each orientation.}
\end{itemize}

\section{Lecture}
The eyes over-exaggerates brightness and contrast. This is because we compare it to the surrounding context and our Center-surround receptive fields respond to the changes.
Using 'high-frequency' we can see details and don't really bother with the surrounding areas, and using 'low-frequency' we see bigger objects and shapes.\\
\subsection{Illusion}
Brightness illusion, with bars, the area just before the down-turn of the specific color, we see a bigger brightness because of the over-exaggeration.\\
\textbf{Hermann Grid:}
The region viewing the intersection is more inhibited than the region of the band going away. Thus the intersection appears darker than the other section. You see dark spots at the intersections of the white bands, but not at the points away from the intersections.\\
\section{Knowledge}
\begin{itemize}
  \item draw and describe how the human eye works.\\
    fovea, cornea, lens, cillary muscles, retina, iris, 
  \item describe the different receptors and how they work.\\
    Rods and Cones;\\
    Rods: sensitive to light, makes it good in dim light scenes. Sees no color and are not in the fovea but everywhere else.\\
    Cones: Sees high-acuity color vision, only present in the middle of the eye(fovea). These also consist of small-medium and large cones which corresponds to being sensitive to the different wavelengths(S/M/L)
  \item discuss difference in human vision between daylight and low-light conditions.\\
    Rods \& cones.\\
  \item explain what is so special about the fovea and what this means for how we use our eyes\\
    It's where we focus images of scenes, where the cones sees high-detail color.
  \item describe the relationship between visual acuity and receptive fields in the retina.\\
    Rods \& cones.\\
  \item explain how a brightness contrast illusion works and why.\\
    Because of Center-surround receptive fields.\\
\end{itemize}

\section{Important notes}
\subsection{Temporal resolution}
Time \& adaptation



\end{document}

