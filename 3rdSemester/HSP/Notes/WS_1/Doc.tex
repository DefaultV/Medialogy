\documentclass{article}
\usepackage[utf8]{inputenc}

\title{Lecture 1 notes and knowledge}
\author{jdrews17}
\date{September 2018}

\usepackage{natbib}
\usepackage{graphicx}

\begin{document}

\maketitle
\newpage


\section{Expected}
Make a brief description of (or addition to) existing VR/AR technology  in the moodle Wiki and describe how it relates to different topics of human senses and perception.  Supply information, links and examples. DELIVERY IN MOODLE WIKI by 15.30.\\

Individual reflections on what you learned IN THE MOODLE WORKSHOP FORUM by 16.15.\\

\underline{Thematic fields}
\begin{itemize}
  \item Resolution (Low-level)\\
    The attention to detail, and the amount of available pixel density in the HMD compared to the receptive fields and the (fovea and retina)
  \item Field of view (High-level)\\
    Stereoscopic depth perception, giving the illusion of depth in the application scene.
  \item Perceiving depth (High-level)\\
    Same as above
  \item Sight (Low-level)\\
    Everything relating to color, brightness, intensity.

  \item Perceiving motion (Low-level)\\
    Within a VR environment with a HMD mounted, you would still be able to feel motion and acceleration.

  \item Proprioception (Low-level)\\
    To know where your limbs and body-parts are specifically without being able to see it. Position of limbs.

  \item Balance (Low-level)\\
    Still counts, as you are wearing a HMD, the application has to follow along the movements of the head and body.

  \item Body movement (Acceleration) (Low-level)\\
    Same as above

\end{itemize}

\section{Knowledge}
Give examples of how knowledge of perception is important for digital media technologies and different modalities.\\

Search for relevant information on topics related to sensation and perception, and also go in depth in the literature to explore further (needed for your mini projects).

\end{document}
