\documentclass{article}
\usepackage[utf8]{inputenc}

\title{Lecture 3 notes and knowledge}
\author{jdrews17}
\date{September 2018}

\usepackage{natbib}
\usepackage{graphicx}

\setlength\parindent{0pt}

\begin{document}

\maketitle
\newpage

\section{Pre-lecture}
Preparation: Read chapter 4\\
\begin{itemize}
  \item \textbf{What are some principles of Figure-Ground organization?}\\
    \textit{Depth, Surroundedness, Symmetry, Convexity, Meaningfulness, Simplicity}
  \item \textbf{What grouping principles are used to identify the doors, windows and wheels of a bus passing by, as part of the same object?}\\
    \textit{Perceptual grouping(Proximity, etc), perceptual interpolation(Ink spots, edge detection)\\Gestalt: Proximity, common fate, similarity}
  \item \textbf{What are some challenges the human brain need to face in identifying objects visually?}\\
    \textit{Gestalt principles, edge detection, using these properly to stitch together the figure}
  \item \textbf{What is meant by bottom-up and top-down information? Can you give examples of when either is used?}\\
    \textit{\textbf{Top-down:} Knowledge and expectations, which affects perception (Expect jon to walk through a door, makes you recognize him faster, and what is more likely to be in a specific scene.)\\\textbf{Bottom-up:} Information contained in the neural signals from the receptors.\\An example can be a wound, seeing it happen (Top-down) differs from feeling it (Bottom-up).}
\end{itemize}

\section{Lecture}
\section{Knowledge}
\begin{itemize}
  \item Give examples of principles for figure-ground organization.
  \item Give examples of different grouping principles used by the visual system
  \item Explain the role of bottom-up and top-down information flow in object recognition
\end{itemize}

\section{Important notes}



\end{document}

