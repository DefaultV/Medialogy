\documentclass{article}
\usepackage[utf8]{inputenc}

\title{Lecture 1 notes and knowledge}
\author{jdrews17}
\date{September 2018}

\usepackage{natbib}
\usepackage{graphicx}

\setlength\parindent{0pt}

\begin{document}

\maketitle
\newpage

\section{Pre-lecture}
\begin{itemize}
  \item What differs between sensation and perception?\\
  Physical conditions of an environment\\
  Organisms with nervous systems can percieve\\
  Sensation refers to initial steps of the process, converting physical features of the environment into electrochemical signals within specialized nerve cells.\\
  Using sensation to form mental representations of the object and events and objects in the environment, and can be stored in memory, usually in humans it also includes conscious awareness of said specific objects.\\
  \item What is a sensory receptor?\\
    Neurons that convert proximal stimuli into neural signals. Eg, photoreceptors(eyes, light), mechanoreceptors(fingers, pressure).\\
  \item What is an action potential?\\
    This signal travels down the neurons Axon, to the axon terminals.\\
    An electrochemical signal that begins in the dendrites of a neuron, that has been stimulated by a signal from another neuron.

  \item What is meant by baseline firing rate?\\
    The firing rate of these action potentials define the changes in the proximal stimuli\\
    A low rate of spontaneous firing at fairly random intervals in the absence of any stimulus.

  \item How does Lidar compare to human senses? Are there similarities? Differences??\\
    No sensory receptors, provide light themselves.
\end{itemize}

\section{Lecture}
\subsection{Pre-face: Perception}
5 quizzes + assignments must be passed for eligible handin of miniproject report.

Individual project: human senses and perception theory in regards to semester project.\\
2-3 peer-reviews of others submissions.\\
Finally; asses your own report with eval.

How? (Guess: Take the specific sensation being utilized in the project and talk about the theory)

\subsection{Nervous system}
(Cerebrum, cerebelium, brainstem and spinal cord)\\
Important stimuli will get recognized by the brain, this also counts for focus, if you focus on an area. Constant unneccesary stimuli is not needed or preferred.\\
Sensory components - Autonomic system - Somatic motor system.\medskip\\
When the electrical charge exceeds a threshold, the neuron will fire an action potential. (\textbf{Conduction signal})\\
Sensory neurons will adapt, change is interesting. If a neuron keeps firing all the time, it will "wash out".\medskip \\
\textbf{Convergence}(Many to one) or \textbf{Divergence}(One to many) E.g. The sense of pain (One neuron can be faulty)\medskip\\
\textbf{Excitatory} or \textbf{Inhibitory} input(Probability of firing) can be collected, this is usually between neurons. The Excitatory neurotransmitter depolarizes the membrane potential and thereby increases the potential that an AP will be generated in the postsynaptic neuron. and the Inhibitory hyperpolarizes it, being the opposite function.\\

Spatial acuity varies across the body, receptive fields (Two-point threshold, distance between two points of receptive fields?)\\

\textbf{Receptive fields}, the area of the retina/skin that produces a change in activity in a ganglion cell. The "Window outside" which can vary in size. The bigger the receptive field, the lower the resolution.\\
A surround inhibition can result in a more fine tuned area in terms of specifics(it acts as a filter if loads of information is being input).\\

\section{Knowledge}
\begin{itemize}
  \item Have a basic understanding of neural communication:\\
  \item Knowledge on excitatory and inhibitory neurons.\\
    \textit{The neurons can either excite or inhibit an AP being fired from the post-synaptic neuron or axon?}\\
  \item Knowledge on receptive fields.\\
    \textit{Determined area of input which produces change in a ganglion cell}\\
  \item explain what is meant by acuity\\
    \textit{Where the receptive field or cell is spefici to the position of the input, e.g. eyesight acuity can see better detail ergo spatial resolution related}\\
  \item explain what is meant by resolution.\\
    \textit{The fine tuning of neurons, in terms of filters and receptive fields, how fine a resolution is in a specific receptive field}\\
\end{itemize}

\section{Important notes}
\subsection{Temporal resolution}
Time \& adaptation



\end{document}

