\documentclass{article}
\usepackage[utf8]{inputenc}

\title{HSP - lecture 7 notes - Attention}
\author{Jannick Drews}
\date{\today}

\usepackage{natbib}
\usepackage{graphicx}

\setlength\parindent{0pt}

\begin{document}

\maketitle
\newpage

\section{Pre-lecture}
Preparation: Read chapter 9\\
\begin{itemize}
  \item \textbf{Why is attention needed? Are there different kinds of attention?}\\
    \textit{Selective attention; divided and focused.
    \begin{itemize}
      \item Filter theory of attention\\
        E.g. Dichotic listening method.
      \item Bottom-up \& top-down attentional control.\\
      \item Value-driven attentional control.\\
    \end{itemize}}
  \item \textbf{What is the cocktail party problem and what does it illustrate?}\\
    \textit{Demonstrates the dependence of awareness and comprehension on attention, e.g. One phenomanon termed \textbf{inattentional blindness}; the failure to perceive a fully visible but unattended visual object. When tasked to focus on one object, the participant can fail to see anything else}
  \item \textbf{What is the difference between inattentional blindness and change blindness?}\\
    \textit{Change blindness; inablity to quickly detect changes in a complex scene. Quickly detecting changes in a complex scene vs Failure to perceive a fully visible but unattended visual object.}
  \item \textbf{What is meant by attentional cuing?}\\
    \textit{Expecting a stimuli, e.g. arrows / sounds indicating something that requires attetion.}
  \item \textbf{What makes a visual search easy or hard?}\\
    \textit{Feature-based attention, e.g. finding a yellow pen on a desk is usually quite easy because it stands out; Feature search, conjunction search.}
  \item \textbf{What is the difference between top-down and bottom-up attentional control?}\\
    \textit{Top down is voluntary control, whereas bottom-up is stimuli driven.}
  \item \textbf{Is it responsible to drive and talk in a mobile phone?}\\
    \textit{Define responsible, but no}
\end{itemize}

\section{Lecture}
\subsection{Attention}
Attention can be modeled; imagine our attention being a spotlight, focusing on information seperately but quickly, this is not always true though.\\
\textbf{Sofias list:}\\
\begin{itemize}
  \item Internal attention (World vs own line of thoughts)
  \item External attention
  \item Selective attention (Processing a subset of available stimuli)
  \item Overt attention (Directing sensory organ to a stimulus)
  \item Covert attention
  \item Divided attention (Being aware of multiple stimuli)
\end{itemize}
\begin{itemize}
  \item Supervisory control\\
    Can be driving.
\end{itemize}
\subsection{Stimuli}
Functional compontents of attention.
\textbf{Salient stimuli} bottom-up stimuli.\\
\textbf{Top-down modulation;} can improve signal to noise ratio for the relevant stimuli.\\
\textbf{Working memory;} identifies the target to focus on.\\
\textbf{Competitive selection;} Helps to eliminate the effect of the distracting stimuli.\\
\subsection{Cueing paradigm, Posner's}
There's Endogenous cues(Top-down) and Exogenous cues (Bottom-up);\\
\subsection{Visual search tasks}
Difficult when e.g. Don't know the packaging or colour or name of a product in a supermarket. Easy when looking for a yellow car in a parking-lot.
\begin{itemize}
  \item Looking for a target in a display containing distracting elements
  \item Examples: Finding your car in a parking lot or a friend in a crowd
  \item Target: The goal of a visual search
  \item Distractor: In visual search, any stimulus other than the target
  \item Set size: The number of items in a visual search display
\end{itemize}
\subsubsection{Conjunction search}
\begin{itemize}
  \item A search for a target defined by the presence of two or more attributes
  \item No single feature defines the target
  \item Defined by the co-occurrence of two or more features
\end{itemize}
\underline{Illusory conjunction}
Mixing multiple features of an element to a new element not previously present.\\
\subsection{Dunno}
Local approach $\rightarrow$ Localizing an individual objects in a scene.
Global approach $\rightarrow$ Identify an entire scene at once.
Attentional blink; "ATTENTIAL UNAWARENESS", refers to how we may be unaware of stimuli in an attended location, if the stimuli occour when we are processing something else.
\section{Knowledge}
\begin{itemize}
  \item Give examples of different kinds of attention.\\
    \textit{Selective attention}
  \item Discuss the role of attention.\\
    \textit{The filter, can't process everything all the time}
  \item Discuss what make a visual search easy or hard.\\
    \textit{Yellow car vs looking for product in supermarket with unknown packaging and name.}
  \item Give examples of what can capture and elude our attention.\\
    \textit{Salient stimuli, (bottom-up), }
  \item Give examples of how knowledge of attention can/should be used in the design of digital media.\\
    \textit{use brain}
\end{itemize}

\section{Important notes}
Bottom-up; Stimulus driven.\\
Top-down; Voluntary.\\ 
4 times higher probabiltiy of missing something while talking on the phone while driving.\\
\end{document}
