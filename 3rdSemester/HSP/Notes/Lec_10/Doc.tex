\documentclass{article}
\usepackage[utf8]{inputenc}

\title{Lecture 9 - HSP}
\author{Jannick Drews}
\date{\today}

\usepackage{natbib}
\usepackage{graphicx}

\setlength\parindent{0pt}

\begin{document}

\maketitle
\newpage


\section{Lecture}
Kinesthetic and proprioceptive feedback can be used to give information to the spacial awareness of your limbs\\
Braille (blindwriting)\\
Tactile belt for pilots etc\\
BrainPort: vision through tactile array on tongue\\
Mechanoreceptors: Feature sensitivity and associated function: check slide\\
Resolving temporal details?\\
Touch 5ms vision 25ms audio 0.01ms\\

\section{Knowledge}
\begin{itemize}
    \item Explain how we use touch to explore our surroundings\\
        \textit{Lateral motion, pressure, static contact, unsupported holding, enclosure, contour following}
    \item Explain of how tactile sensitivity and acuity varies for different (major) body parts.\\
        \textit{Sensory humonculus}
    \item Discuss important things to consider when designing tactile feedback in interactive applications.\\
        \textit{Haptic exploration, what we can perceive in acuity, which bodyparts are available}
    \item Name at least one kind of somatosensation that does not involve tactile information.\\
        \textit{Pain, temperature, proprioceptive feedback}
    \item Give examples of how we can experience ownership of virtual or additional body parts\\
        \textit{Rubber hand illusion, brain adapts to the new tactile information and adapts body parts with the same visual input}
    \item Explain how the vestibular sense works and how it can be disrupted.\\
        \textit{vestibulo-ocular reflex: information from the Vestibular sense is used to compensate the eye movements so that the world is seen more "stable", constants will always lead to adaptation, and disruption is the reverse of it instantly}
    \item Give examples of induced self motion.\\
        \textit{Vection, (train illusion when moving), stopping a movement}
    \item Describe the relationship between eye movements and the vestibular sense with some examples.\\
        \textit{vestibulo ocular relfext, information about hand and eye movement to keep stability}
\end{itemize}

\section{Important notes}


\end{document}
