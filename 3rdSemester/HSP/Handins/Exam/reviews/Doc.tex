\documentclass{article}
\usepackage[utf8]{inputenc}

\title{Human Senses and Perception Exam-handin}
\author{Jannick Drews}
\date{\today}

\usepackage{natbib}
\usepackage{graphicx}
\usepackage{pdfpages}
\setlength\parindent{0pt}

\begin{document}

\maketitle
\newpage

\section{My essay}
\includepdf[pages=-,pagecommand={},width=2\textwidth]{../../Mini-Project/HSP-Drews-20176027.pdf}

\section{Reviews}
\small
\subsection{Review of Nick Boe Elsborg; Vision and Hearing as Perceptual Systems - Similarities and Differences}


Good \textbf{introduction} and elaboration of what the essay will cover.\\
The \textbf{theory section} mentions "events", this can be unclear for the reader, a potential edit could just be to write shortly what the event is in regards to. e.g; "events, e.g. a person walking or rock colliding with a surface".\medskip \\
In \textbf{section 2.2}, paragraph one, "aid in its identification and/or categorization", this can be unclear to the reader in regards to what the author meant. What is meant by identification and categorization? A solution could be to shortly cover what the words are in regards to.\\The section is general is very nicely ordered, explained and concise. Good refs and statements.\medskip \\
\textbf{Section 2.4}, concise explanations of what frequency, amplitude and waveform is, might be a bit squished or rushed explanation but the authors point can be understood.\\First sentence in \textbf{2.4.1} seems out of place, either could be deleted or rephrased, Paragraph 2 is missing a ref for the statement but else it's a good explanation and also good comparison between how the two ears can relate more or less to how the two eyes function in terms of depth. Paragraph 4 could also use a reference or two for the statements on doppler effect.\\\textbf{Section 2.4.2}, even though the abstract words have footnotes, it still needs references for the statements made. Else the section is quite nice, fact heavy but the author clearly tried to keep it simple whilst also giving off the understanding properly.\medskip \\
In \textbf{section 2.5}, first sentence is very odd and could potentially seem out of place for the reader, a potential edit could be; "[...] which is then converted to X representation, that COMPARES it with the brains memory".\\It's although very clear the author has given both topics thought when comparing sounds to visual, this is in regards to the EST comparsion when listening to footsteps instead of seeing it visually.\medskip \\
In \textbf{section 3, the discussion}, the author provides good claims of how changing related features of a certain object in regards to its sorroundings in a digital scene, could have drastic effects on how the person will perceive the object compared to how their usual expectations were for said object.\\It's also very nice that the Author shows thought in relation to his semester project, that inhibiting the visuals of his prototype could increase the focus on the auditory aspects of it when a user uses it.\\Overall the discussion is good, but paragraph 3 especially seems a bit out of place and could potentially be read as more of a fact than a discussion. Either explain the others in more detail or drop paragraph 3.\medskip \\
\underline{Overall}\\
Very few gramatical mistakes, and none were sufficient enough to make the user unable to understand what was being told by the author, the refs are very nicely done and the sources are credible correct where needed, especially with pagenumbering and the factual content is exhaustive and accurate. The essay can seem pressured in some sections due to word limitations, as the author was close to exceed those limits in small occurances in the essay. The reader can easily understand what is being relayed through the topics.\\
In terms of learning goals, all are covered which relates to the semester project of the author, you could discuss whether or not top-down and bottom-up information, masking or how the auditory system specifically detects sounds are really needed to cover the semester project, but that I would argue is not needed.\\
Also plus points for writing the essay in \LaTeX


%Is the factual content correct?
%Is the report of adequate length? (2000-2500 words excluding references)  YES
%Does the author demonstrate knowledge of the topics corresponding to the learning goals? YES
%Is the content organized and presented in such a way that a reader can follow the reasoning? YES
%What sources are used and are they cited and referenced properly? YES
%////////////////////////////////



\newpage
\subsection{Review of Mathias Friis Jensens; Human depth perception and the Kinect depth sensor}

Section \textbf{"Vision"}; explanation of how vision occours is fine, although the end of paragraph 1 is missing a reference to the statement made in regards to sensors. Also the author makes it sound like the light is captured by pixels which are the RGB-values. This is incorrect and a potential edit could be; "The light is captured by the cells in the sensor, which are then converted to a digital image by a converter."\\Paragraph 2 is also missing a reference, in regards to how we perceive depth in the world. Good transition from one section to the next, but very extensive and not neccesarily needed, e.g. a potential edit could be to shorten the paragraph to "There are many similarities between a camera and an eye, the depth perception is although very unique in this situation and will therefore be compared to other system in the next section.".\\

Section \textbf{"Depth perception through eye and sensor"}; It is hard to check the factual statement made about "Convergence", since it is missing a reference to the material explained. Also, as explained in the course book; the angle between the eyes increases as the objects distance from the eye decreases. The explanation from the author could potentially confuse the reader; and edit could be "the angle between the eyes will get bigger, as the object moves closer to the eyes".\\

Section \textbf{"Binocular depth cues"}; The very first statement can potentially be confusing or misleading for the reader, as it seems the author claims that the only functionaility needed in the eye for a retinal image, is the lens which is incorrect. A potential fix could just be to delete this statement made whilst modifying the next sentence to fit the elaboration of the follow-up.\\

Section \textbf{"Stereograms"}; This section does cover the explanation of anaglyphs, the factual content is understandable for the reader but is weighed down heavily because of the missing credibility which would have been provided through a reference or citation.\\

\underline{Overall}\\
A lot of statements made in the essay are not properly referenced or completely missing the references. Examples can be seen in the Vision and Depth perception through eye and sensor sections. The author makes multiple factual statements which are never referenced or the whole section only has one reference where the author is explaining multiple things at a time.\\Some factual statements that are made in the essay wary or is unclear in regards to it validity. The author occasionally uses specific terms which are never explained in enough detail to make sense of it, the reader can potentially delve into a very confusing statement because of the lacking explanations.\\ The discussion is fine and compares the advantages and disadvantages of the topics in relation to the semester project. Although most of the elements stated are already covered throughout the rest of the essay and could arguably be redundant.\\The author clearly did research most of the topics he described, but the lacking citations and references made the validity of the statements fallible.\\Sidenote; very odd way of referencing the figures presented in the essay, else the sources, where given, are correct.
% Over 2k
%Is the factual content correct?
%Is the report of adequate length? (2000-2500 words excluding references)  YES
%Does the author demonstrate knowledge of the topics corresponding to the learning goals? YES
%Is the content organized and presented in such a way that a reader can follow the reasoning? YES
%What sources are used and are they cited and referenced properly? YES
%////////////////////////////////

\newpage
\section{Self-reflection on the essay and the reviews}
\subsubsection{Response to Author 1}
The thresholding section, pointed out by Author 1, does indeed seem out of place and may be too short for having any relevance to the rest of the detailed topics elaborated on beforehand.\\

Author 1 explains how more sources could have been used to support the factual content of the essay. This could indeed help the credibility of both the original author and relay the factual content with more certainty. Although you could argue whether it is indeed needed for relaying the goal of this essay.\\

I must also concur with the description given of edge completion. That section would definitely benefit from having a figure to illustrate what is meant, as the reader can have potential trouble imagining everything the author meant.

\subsubsection{Response to Author 2}
Author 2 makes the claim that the report is missing a discussion section which would elaborate on comparison to media-technology, and that the report is \textit{"pure theory"}. If Author 2 had read the section introduction of Section 2 Theory and comparisons, Author 2 would know that the human senses and perception theory were to be explained alongside a comparison and discussion of the media-technological counterpart iteratively.\\Author 2 is also incorrect in stating both that it is a report and that the content is \textit{"pure theory"}. The assignment is an essay, and the essay does indeed hold fundamental facts.\\

Author 2 also makes a note that the theme is not very apparent in the essay. This does make sense if you are looking for a comparison between one Human senses and perception topic against a media-technological counterpart, although this essay does indeed cover multiple areas, hence the author of the essay may have been confused as to what to specifically call the essay.\\

Small but still noteworthy is that the description of what a "visual system" is, could potentially have aided in helping the reader understand what was being relayed, in that section of the essay.\\

Lastly, author 2 also mentions that some learning goals are missing, this could potentially be open for discussion, whether or not all the human perceptual systems are relevant to the original authors semester project, as it revolved only around detecting hand postures given in a scene.\\

\underline{Overall}\\
There seem to be a few holes in the essay, in relation to descriptions of scientific terms used, such as "Visual system". Else the essay seems fulfilling and relevant. The essay also contains credible sources, which are all relevant to the respective topics.

\end{document}
