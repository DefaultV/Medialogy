\documentclass{article}
\usepackage[utf8]{inputenc}

\title{Notes Lecture 7 HSP}
\author{Jannick Drews}
\date{\today}

\usepackage{natbib}
\usepackage{graphicx}

\setlength\parindent{0pt}

\begin{document}

\maketitle
\newpage

\section{Pre-lecture}
Preparation: Read chapter 10\\
\begin{itemize}
  \item What is a periodic sound wave and can you give some examples of periodic sounds?\\
    \textit{Waves, in which the cycles of compression and rarefaction repeat in a regular or periodic fashion, as opposed to aperiodic waves (Turbulent events, e.g. door slam), and example of a periodic sound wave can be a tuning fork. The most simple periodic sound wave is a \textbf{pure tone}, in which the wave corresponds to a sine-wave in the mathmatical formula context.}
  \item What is a complex sound?\\
    \textit{A sound where the air-pressure of the sound-wave over time is complex or very irregular???}
  \item What is the physical measure (closest) corresponding to perceived pitch?\\
    \textit{Frequency}
  \item What is the physical measure (closest) related to a sound's loudness?\\
    \textit{Amplitude}
  \item What do the equal loudness curves tell us?\\
    \textit{All the tones specified in the curve, are equally loud, even though the dB SPL(Amplitude) and Frequency(Hz) differs.\\More info: Sensitivity over frequency of tones, E.g. a 10 dB louder frequency wave but is on a frenquency of 0.1, sounds just as loud as a 1kHz frequency on a dB of 50}
  \item What is shown in a frequency spectrum?\\
    \textit{The frequency spectrum of an electrical signal is the distribution of the amplitudes and phases of each frequency component against frequency.\\(https://www.collinsdictionary.com/dictionary/english/frequency-spectrum)}
  \item What are the ossicles and why do we need them?\\
    \textit{The three smallest bones in the human body, they transmit sound energy from the tympanic membrane to the inner ear. (Malleus, Incus, Stapes), it's specifically the Malleus that is connected to the tympanic membrane.}
  \item What is the basilar membrane doing to help differentiate high and low frequencies?\\
    \textit{The basiliar membrane seperates out frequencies of the sinsoidal components of a complex wave, the stiff base of the membrane responds most readily to high frequencies, whereas the apex responds to low frequencies because it's more flexible.}
  \item What are the inner and outer hair cells and what do they do?\\
    \textit{The inner haircells are responsible for transducing sound into neural signals, and the outer hair cells serve to amplify and sharpen the responses of the inner hair cells. Both are crucial parts of the corti, along with the tectorial membrane.}
  \item How is phase synchronization used in auditory signalling?\\
    \textit{A phase refers to the time-displacement of one wave, e.g. two waves in different phase has two different "locations" on a graph, either + or -. Something about frequency hitting at the same time as the AP from neural signals ??? }
\end{itemize}

\section{Lecture}

\section{Knowledge}
\begin{itemize}
  \item Explain what sound is and how the audiotory system detects it\\
    \textit{Pressure-waves (vibrations) of specific amplitude and frequency. Only through matter. }
  \item Describe some basic operating characteristics of the auditory system (Audibility thresholds, perceived loudness for different frequency ranges).
    \textit{Audibility threshold; Variations over the span of frequencies, it depends on the frequency.\\Perceived loudness for different frequency ranges; e.g. low vs high frequency needs a dB boost to make it equal in loudness}
  \item Give examples of sounds that have a defined pitch
    \textit{Vowels, A whistle, tuning fork. Guitar string; defined pitch is a periodic wave. all periodic wave have a harmonic spectrum, which gives rise to a defined pitch. Examples of sounds that do not give defined pitch; white noise, traffic noise, noise. Something like a drum is hard to agree on a tune for.}
  \item Give examples of when masking occours
    \textit{Temporal masking; one loud sound and then a short sound afterwards, the loud tone can mask the weaker one.\\the same with instruments e.g. basoon masking a piccolo if the basoonist play loudly.}
\end{itemize}

\section{Important notes}
\begin{itemize}
  \item What is sound?\\
    \textit{Movement that disturbs air-molecules, which then collides with other molecules, this then causes changes in air-pressure which propegates outwards.}
\end{itemize}
Pure tone has one frequency. Which is related to sin-wave or cos-wave\\
Relationship between sound speed:\\
$ v = f\cdot\lambda$\\
Where v is the m/s, f is the frequency and lambda is the meters\\
All sound waves can be described with the sinoids\\
Combining a high-amp low-freq wave with a reverse feature wave, gives a period? Which we perceive as two tones. But the combined tone will have a fundamental parent, either one of the waves having the same cycle.\\
Complex waveforms are usually square in a graph because it's very detailed in the waveform.\\
How can we utilize auditory masking? Compression and Audio coding.
\end{document}
