\documentclass{article}
\usepackage[utf8]{inputenc}

\title{Lecture 9 - HSP - Notes}
\author{Jannick Drews}
\date{\today}

\usepackage{natbib}
\usepackage{graphicx}

\setlength\parindent{0pt}

\begin{document}

\maketitle
\newpage

\section{Pre-lecture}
Preparation: read chapter 11(skip or skim the section auditory brain), read chapter 12(skim "the sounds of speech" and perceiving sounds of speech), skip brain pathways for speech perception and production, skim sections on melody, scales and music perception\\
\underline{Chapter 11}
\begin{itemize}
    \item \textbf{What auditory cues are used to localize sounds?}\\
        \textit{Inter-aural time difference (ITD), the arrival time to each ear, Interaural level difference (ILD), Pinna and Head cues (Asymmetrical form and shape, helps us localize along the vertical axis), amplitude, reverb(reflection)}
    \item \textbf{What is the cone of confusion and which cue does it relate to?}\\
        \textit{ITD, a cone of a specific area outwards from the ear, where it's hard to pinpoint sound location}
    \item \textbf{What is the precedence effect?}\\
        \textit{The precedence effect or law of the first wavefront is a binaural psychoacoustic effect. When a sound is followed by another sound separated by a sufficiently short time delay, listeners perceive a single auditory event; its perceived spatial location is dominated by the location of the first-arriving sound.}
    \item \textbf{What is auditory scene analysis?}\\
        \textit{In psychophysics, auditory scene analysis is a proposed model for the basis of auditory perception. This is understood as the process by which the human auditory system organizes sound into perceptually meaningful elements.}
    \item \textbf{Give examples of how gestalt grouping principles can be applied in the auditory domain}\\
        \textit{Proximity, similarity. (Pitch \& Frequency), if tones are close in frequency, they can be perceived as related, whereas being far apart in frequency makes them seperate, common fate (onset) might be perceived as coming from the same sound source}
\end{itemize}
\underline{Chapter 12}
\begin{itemize}
  \item \textbf{How does the spectrum of a speech vowel typically look?}\\
    \textit{.}
  \item \textbf{What is categorical perception and can you give an example for speech?}\\
    \textit{.}
  \item \textbf{What is an octave and what is special about how we perceive it?}\\
    \textit{All being multiple of the fundamental frequency, has a defined pitch?}
\end{itemize}


\section{Lecture}

\section{Knowledge}
\begin{itemize}
    \item Explain how and why the most important auditory localization cues work\\
        \textit{ITD(time difference), ILD(), Intensity(Difference between two ears), Shape of ears, head related and frequency dependent (High frequency can give acoustic shadows)}
    \item Explain what auditory scene analysis is and how it helps to solve the cocktail party problem\\
        \textit{Timbre, pitch, focus, and gestalt principles, segregate and follow an auditory stream(Cocktail party problem)}
    \item Give examples of how gestalt rules are used in auditory stream segregation\\
        \textit{look above}
    \item Give examples of how spectrum of typical speech sounds look\\
        \textit{.}
\end{itemize}

\section{Important notes}


\end{document}
