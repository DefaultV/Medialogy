\documentclass{article}
\usepackage[utf8]{inputenc}

\title{Lecture 4 notes and knowledge}
\author{jdrews17}
\date{September 2018}

\usepackage{natbib}
\usepackage{graphicx}

\setlength\parindent{0pt}

\begin{document}

\maketitle
\newpage

\section{Pre-lecture}
Preparation: Read chapter 5\\
\begin{itemize}
  \item \textbf{What is meant by spectral reflectance?}\\
    \textit{The molecular structure of an objects surface, whilst the percieved color of objects depends on the SPD(Spectral Power Distribution) and also how that specific thing reflects the light(color constancy). \textbf{The proportion of light that an objects surface reflects rather than absorbs.}}
  \item \textbf{What are the three perceptual dimensions of colour}\\
    \textit{Hue, saturation and Brightness(HSV), which is also the \textbf{perceptual} experience of colours}
  \item \textbf{What is meant by univariance? When do we experience it?}\\
    \textit{It's nearly impossible to backtrace to determine the wavelength of light that caused a response in a single cone. This is expressed by the principle of univariance. Since cones can be significantly different in their response to different amounts of light, you might not know how much of a specific light was the causation of the response, "in other words" the strength of the response generated by a cone when it transduces light depends only on the \textbf{amount of light} transduced. Not the wavelength. Hence why we can perceive some different lights as the same. More to note from the boom \textbf{"The principle of univariance means that colour vision depends crucially on the relative response of multiple cone types." The strength of the response of a cone when it transduces light depends on the amount of light transduced, not the wavelength}}
  \item \textbf{How does the spectral sensitivity function differ between photoreceptors?}\\
    \textit{As given in Lecture 2, a photoreceptors main function is to transduce light into neural signals, (rods, and cones). The spectral sensitivity \textbf{is the probability that a photon of light with any given wavelength will be absorbed by a cone's photopigment}, although cones can be sensitive to their respective (S,M,L) wavelengths, they are all capable of absorbing more than just their narrow sensitive spectrum. Which is how they differ.}
  \item \textbf{How is oppocency represented in the retina?}\\
    \textit{The current view of colour-vision is a two-stage process, first the trichromatic representation, and then the oppocent color representation(What happens after the point of transducing by the photoreceptors). Which is how the RGC's and color-selective neurons in the brain process the cone signals.(which is what gives the yellow colour it's spot on the pedestal over primary colours(colors are two opponent pairs, red-green, blue-yellow)). Which have made way for an interesting approach; \textbf{specific circuits in the retina are though to work like the following;}
    \begin{itemize}
      \item +S-ML Circuit
      \item +ML-S Circuit
      \item +L-M Circuit
      \item +M-L Circruit
    \end{itemize}
    Where the first letter corresponds to above baseline firerate at the specific wavelength, and the other letters to below baseline firerate at the specific wavelength.
    }
  \item \textbf{What is meant by colour constancy? Can you give an example of colour constancy?}\\
    \textit{Even two lightsources with different SPDs shine on an object, we can base our perception of the colour of an object on the reflectance of the object, and not on the SPD of the light reflected, is called the \textbf{color constancy}. Paper is one good example given by the book. Maybe leather?}
\end{itemize}

\section{Lecture}


\section{Knowledge}
\begin{itemize}
  \item Explain why perceiving color is beneficial for humans.\\
    \textit{Primates; Distinguish between berries and shit(dangerous explanation), seeing health, on an infant or other persons. See rot on bread.}
  \item Explain the theory of trichromatic colour vision.\\
    \textit{Look above}
  \item Explain the opponent process theory of colour vision.
    \textit{Look above}
  \item Discuss similarities and differences between computer and human colour vision.
    \textit{Color channel, adaptation, trichromatic}
\end{itemize}

\section{Important notes}
\subsection{Opponent theory vs Trichromatic theory}
Opponent theory; Cones respond opposite ways to different wavelengths of the visible spectrum, basically there's 4 basic colours in 2 pairs, green-red, blue-yellow.\\
Trichromatic theory; Cones are sensitive to their respective photopigment, and thereby color. Which there are 3 of.\\


\end{document}

