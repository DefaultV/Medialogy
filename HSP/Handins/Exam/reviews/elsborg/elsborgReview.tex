\documentclass{article}

\begin{document}

\section{Review of Nick Boe Elsborg; Vision and Hearing as Perceptual Systems - Similarities and Differences}


Good \textbf{introduction} and elaboration of what the essay will cover.\\
The \textbf{theory section} mentions "events", this can be unclear for the reader, a potential edit could just be to write shortly what the event is in regards to. e.g; "events, e.g. a person walking or rock colliding with a surface".\medskip \\
In \textbf{section 2.1}, concise and fullfilling explanation of how figure-ground works. Potential reader confusing in paragraph 3, when talking about perceiving occluded objects, is this in regards to edge completion or "interpolation" as mentioned above or the good continuation principle? Also, when mentioning the surroundedness and meaninfullness, either explain shortly what they are or remove the sentence and say they will not be covered since they are not related or relevant.\medskip \\
In \textbf{section 2.2}, paragraph one, "aid in its identification and/or categorization", this can be unclear to the reader in regards to what the author meant. What is meant by identification and categorization? A solution could be to shortly cover what the words are in regards to.\\The section is general is very nicely ordered, explained and concise. Good refs and statements.\medskip \\
In \textbf{section 2.3}, "But how do we distinguish between events? [...]", nice giving motivation to the reader and putting up some valid questions. Although it's sadly short lived, since only the second question "When does one start and another end?" was answered(as I could understand). Section is nice for explaining what the events are, it could potentially seem a bit short or rushed, but the point comes across fine.\medskip \\
\textbf{Section 2.4}, concise explanations of what frequency, amplitude and waveform is, might be a bit squished or rushed explanation but the authors point can be understood.\\First sentence in \textbf{2.4.1} seems out of place, either could be deleted or rephrased, Paragraph 2 is missing a ref for the statement but else it's a good explanation and also good comparison between how the two ears can relate more or less to how the two eyes function in terms of depth. Paragraph 4 could also use a reference or two for the statements on doppler effect.\\\textbf{Section 2.4.2}, even though the abstract words have footnotes, it still needs references for the statements made. Else the section is quite nice, fact heavy but the author clearly tried to keep it simple whilst also giving off the understanding properly.\medskip \\
In \textbf{section 2.5}, first sentence is very odd and could potentially seem out of place for the reader, a potential edit could be; "[...] which is then converted to X representation, that COMPARES it with the brains memory".\\It's although very clear the author has given both topics thought when comparing sounds to visual, this is in regards to the EST comparsion when listening to footsteps instead of seeing it visually.\medskip \\
In \textbf{section 2.6}, the statement; "if the communication platform changes (eg. from speech to email) our visual sense can become increasingly more superior." is missing a reference since its a factual statement. Else all the examples are nicely explained and refers to their respective sound or visual cues, whilst also the idea of the section comes across nicely and feels relevant when doing a comparison of the systems.\medskip \\
In section 2.7, the explanation of the project is fine, and the reader gets the idea of why all the auditory and visual topics were presented in the first place, this could potentially have been one of the very first sections in the essay to make it more clear why the reader is delving into the perceptual topics(ultimately giving the reader a reason for why he/she is reading about the topics in the first place).\medskip \\
In \textbf{section 3, the discussion}, the author provides good claims of how changing related features of a certain object in regards to its sorroundings in a digital scene, could have drastic effects on how the person will perceive the object compared to how their usual expectations were for said object.\\It's also very nice that the Author shows thought in relation to his semester project, that inhibiting the visuals of his prototype could increase the focus on the auditory aspects of it when a user uses it.\\Overall the discussion is good, but paragraph 3 especially seems a bit out of place and could potentially be read as more of a fact than a discussion. Either explain the others in more detail or drop paragraph 3.\medskip \\
\underline{Overall}\\
Very few gramatical mistakes, and none were sufficient enough to make the user unable to understand what was being told by the author, the refs are very nicely done and the sources are credible correct where needed, especially with pagenumbering and the factual content is exhaustive and accurate. The essay can seem pressured in some sections due to word limitations, as the author was close to exceed those limits in small occurances in the essay. The reader can easily understand what is being relayed through the topics.\\
In terms of learning goals, all are covered which relates to the semester project of the author, you could discuss whether or not top-down and bottom-up information, masking or how the auditory system specifically detects sounds are really needed to cover the semester project, but that I would argue is not needed.\\
Also plus points for writing the essay in \LaTeX


%Is the factual content correct?
%Is the report of adequate length? (2000-2500 words excluding references)  YES
%Does the author demonstrate knowledge of the topics corresponding to the learning goals? YES
%Is the content organized and presented in such a way that a reader can follow the reasoning? YES
%What sources are used and are they cited and referenced properly? YES
%////////////////////////////////
\end{document}
